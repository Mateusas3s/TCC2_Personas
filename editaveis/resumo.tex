\begin{resumo}

As personas podem ser usadas no processo de design de interface em um sistema digital. Elas são personagens fictícios, que representam usuários reais de algum produto. Profissionais de Design e da Engenharia de Software podem utilizá-las, principalmente tratando-se do desenvolvimento centrado no usuário. Na academia abordagens inovadoras têm sido exploradas para incentivar e engajar os alunos a aprenderem os conteúdos disciplinares. Os jogos são uma destas abordagens e que tem se tornado populares no meio de ensino da Computação. Dado o contexto, o objetivo deste trabalho é propor um elenco de personas que auxilie o desenvolvimento de jogos digitais para aprendizagem em Interação Humano-Computador (IHC). Para isto foi feita uma revisão na literatura, foi conduzido um \textit{survey} e utilizado um método para construção de personas. Como resultado, é proposto um elenco com quatro personas. Estas personas representam a amostra coletada no \textit{survey}, priorizando as características mais frequentes que foram analisadas. Numa visão geral, as personas elaboradas neste trabalho representam jogadores que por conta de seu pouco conhecimento em IHC, usariam jogos principalmente para aprender algum conteúdo novo. Estes têm expectativas de que o jogo tenha um design atraente, \textit{feedbacks}, regras fáceis de entender e uma mecânica de jogo fácil aprender. Além disto, almejam ter uma experiência de satisfação em aprender jogando, de confiança na aprendizagem através do jogo, de percepção da relevância do conteúdo apresentado no jogo, diversão, desafio e de ter a sua atenção focada enquanto jogam.


\vspace{\onelineskip}
    
\noindent
\textbf{Palavras-chaves}: Personas. Jogos Digitais. Jogos para Aprendizagem. Jogos Sérios. Interação Humano-Computador. \textit{Survey}.
\end{resumo}
