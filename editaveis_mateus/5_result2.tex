\chapter{Elenco de Personas}
\label{chap:ele-pers}

Neste capítulo é apresentado o elenco de personas. Este é composto pela persona primária, Victor Matheus Farias (Seção \ref{sec:p1}); a persona secundária, Afonso de Souza Queiroz (Seção \ref{sec:p2}); a persona suplementar, Natália Figueiredo (Seção \ref{sec:p3}); e a anti-persona Rafael Medeiros (Seção \ref{sec:p4}).


\section{Persona Primária}
\label{sec:p1}

A persona primária representa o jogador-alvo, aquele que o design do jogo deve atender com prioridade. A Tabela \ref{tab:Table_persona1} apresenta os atributos desta persona.

\begin{table}[htbp]
\centering
\caption{Persona Primária}
\label{tab:Table_persona1}
\small
\begin{tabular}{| m{0.25\textwidth} m{0.65\textwidth}|}
\hline \multicolumn{2}{|c|}{\textbf{Identidade}} \\ \hline
& \\

\begin{center} 
\includegraphics[scale=0.06]{figuras/personas/portrait-3353699_1920.jpg} 
Fonte: Pixabay\tablefootnote{https://pixabay.com/photos/portrait-people-adult-man-face-3353699/}
\end{center} 



&

\textbf{Nome: } Victor Matheus Farias

\textbf{Idade:} 19 anos

\textbf{Ocupação:} Estudante de Engenharia de Software na UnB, Faculdade do Gama.

\\ \hline


\multicolumn{2}{|c|}{\textbf{Descrição}} \\ \hline
\multicolumn{2}{|p{15cm}|}{
    \begin{tabular}[c]{@{}l@{}}\\
    Aprender um conteúdo novo é meu o principal objetivo ao usar jogos educacionais.\\ Atualmente eu uso alguns, mas com uma frequência moderada, não gosto de gastar\\ muito tempo com jogos. Vejo-os como ferramentas que me auxiliam no processo de \\aprendizagem. Estou cursando a disciplina de IHC e não tenho muito conhecimento \\em relação a elaboração de design de interfaces e o pouco que tenho está somente no \\âmbito disciplinar. Desejaria utilizar um jogo que me ajudasse a aprender o conteúdo.\\ 
    \\
    Geralmente quando vou estudar ou sanar alguma dúvida que tenho sobre o conteúdo \\eu pesquiso na internet, em alguns casos assisto vídeo aulas e também utilizo do \\material disponibilizado pelo professor.\\
    \\
    Um jogo educacional pra mim, deve \textbf{responder bem às minhas ações}, me ensinando\\ caso eu erre por exemplo; deve ter um \textbf{design legal}, talvez com um tema; deve ser \\\textbf{simples de se aprender a jogar}, não tendo regras extensas e tutorias longos; e \\também deve ser \textbf{fácil de jogar}. \\
    \\
    E por fim o que eu espero de um jogo educacional é sentir \textbf{satisfação}, aprender com um\\ pouco de \textbf{desafio} e \textbf{diversão} é muito bom. Ter a percepção da \textbf{relevância} do assunto\\ que estou aprendendo é algo que me dá \textbf{confiança} que vou atingir meu objetivo de\\ estudo. E mesmo não gostando de gastar muito tempo com jogos, mas se eu sentir que \\estou sendo produtivo com certeza vou me manter \textbf{focado}.\\ \\
    \end{tabular}
} \\ \hline
\end{tabular}
\end{table}
\newpage

Assim como na amostra, a persona Victor Matheus Farias (Tabela \ref{tab:Table_persona1}) é caracterizada por um jovem de 19 anos do sexo masculino, que cursa Engenharia de Software na UnB. A persona primária reflete a parcela mais significativa da amostra, que contempla os alunos que desejam pelo menos aprender algum conteúdo através dos jogos e que estão cursando a disciplina de IHC (31\% da amostra). Isto não descarta os jogadores que também desejam usar esse tipo de jogo para avaliar seu conhecimento ou para revisar o que já aprenderam (persona secundária e suplmentar, apenas há um foco maior para que o jogo esteja voltado a auxiliar o aluno a aprender um conteúdo novo, enquanto que os outros objetivos de estudo são menos relevantes. 

Destes que cursam IHC, acabam sendo refletidas na persona primária aqueles que tem experiência de design em disciplinas acadêmicas, mesmo antes de cursar IHC, quanto em outros meios (21\% da amostra). Isto demonstra que o jogo pode abordar conceitos mais teóricos e fundamentais de IHC, pois os jogadores não teriam até então uma educação formal quanto a IHC.

Ainda outra parcela significativa que esta persona representa, está relacionada àqueles que ainda usavam jogos para aprendizagem (31\% da amostra). Estes revelaram ter uma frequência moderada no uso deste tipo de jogo. Ou seja, já que o público-alvo tem uma afinidade com jogos para aprendizagem, um jogo voltado para um conteúdo em IHC deveria replicar características de jogos que estes jogadores apreciam, como o Duolingo e Perguntados.

Pode-se perceber também que a persona primária representa como um todo a amostra, quando relacionada aos meios utilizados pelos alunos para sanar suas dúvidas, usando meios não pessoais (internet, video aulas e livros da disciplina) como prioridade, mas não descartando os meios pessoais (colegas, monitores e professores). Isto destaca que o jogo poderia ter meios nos quais o jogador pudesse ser direcionado à materiais mais completos sobre o assunto em IHC, como artigos, livros digitais etc.

Em relação aos requisitos de qualidade em jogos, apreciados pelos jogadores, a persona primária reflete aqueles que tiveram maior pontuação (Tabela \ref{tab:req-pont}), como Design Atraente, Feedback ao Jogador, Regras Fáceis e Facilidade em Aprender. Os outros requisitos também tem seu peso dentro do jogo, mas com menor relevância, no caso eles são evidenciados nas personas secundária e suplementar tendo em vista que não são os mais importante para o público-alvo, exceto os três últimos (ponto e recompensas; ranking dos jogadores; e narrativa), por estarem com uma pontuação bem mais baixa em relação aos outros requisitos.  

E sobre as experiências esperadas pelos jogadores são destacadas novamente as de maior pontuação (Tabela \ref{tab:exp-pont}), como a Satisfação em jogar e aprender, Confiança na aprendizagem através do jogo, Percepção da Relevância do Conteúdo, Diversão, Desafios e a Atenção Focada nos Jogos. Os dois últimos (o jogo ser o principal meio de aprendizagem e a interação com outros jogadores) foram excluídos por apresentarem uma pontuação muito mais baixa em comparação às outras experiências. 

\section{Persona Secundária}
\label{sec:p2}

A persona secundária representa alguns respondentes certas diferenças quando comparada com a persona primária, mas ao ser contemplada agrega ao jogo elementos que podem atrair mais usuários. A Tabela \ref{tab:Table_persona2} apresenta os atributos desta persona.


\begin{table}[htbp]
\centering
\caption{Persona Secundária}
\label{tab:Table_persona2}
\small
\begin{tabular}{| m{0.25\textwidth} m{0.65\textwidth}|}
\hline \multicolumn{2}{|c|}{\textbf{Identidade}} \\ \hline
& \\

\begin{center} 
\includegraphics[scale=0.06]{figuras/personas/model-2911332_1920.jpg} 
Fonte: Pixabay\tablefootnote{https://pixabay.com/photos/model-businessman-corporate-2911332/}
\end{center} 

&

\textbf{Nome: } Afonso Souza de Queiroz

\textbf{Idade:} 19 anos

\textbf{Ocupação:} Estudante de Engenharia de Software na UnB, Faculdade do Gama

\\ \hline


\multicolumn{2}{|c|}{\textbf{Descrição}} \\ \hline
\multicolumn{2}{|p{15cm}|}{
     \begin{tabular}[c]{@{}l@{}}\\
        Já joguei alguns jogos educacionais onde meu principal objetivo era aprender um\\ conteúdo novo. Além disso era interessante quando o jogo possibilitava que eu revisasse\\ o conteúdo e avaliasse o que tinha aprendido. Eu jogava com certa moderação. Por mais\\ que atualmente eu não esteja usando algum jogo educacional, eu parei de jogar pois \\alcancei meus objetivos de estudo e me sentiria da mesma forma se houvesse algum\\ jogo que me ajudasse nas matérias da faculdade.\\ 
        \\
        Ainda não fiz a disciplina de IHC, por isso não tenho muito conhecimento em relação\\ a elaboração de design de interfaces, sendo que o pouco que tenho veio de projetos de\\ outras disciplinas e atividades extra curriculares.\\
        \\
        Gosto mais de estudar sozinho, pesquisando na internet e indo por conta própria atrás\\ de materiais de estudo. Apenas quando não acho nada relevante, recorro aos materiais\\ disponibilizados pelo professor.\\
        \\
        Para mim, um jogo onde vou aprender deve ter um design com um \textbf{padrão simples},\\ mas que desperte \textbf{interesse}; deve me dar bons \textbf{feedbacks} a cada interação que faço\\ no jogo; deve ser algo \textbf{simples de aprender como jogar} e \textbf{simples de jogar}; e caso\\ tenha textos sobre o conteúdo, seria bom que os textos fossem bem objetivos e com\\ uma cor e fonte que facilitasse a leitura.\\ 
        \\
        Espero ter a sensação logo de cara que \textbf{não vou perder meu tempo} com o jogo, mas\\ que ao final vou perceber o quanto foi \textbf{importante aprender o conteúdo} e o quanto\\ foi \textbf{satisfatório} ter me \textbf{divertido} aprendendo. E espero também de certa forma ser\\ \textbf{desafiado}, pois isso me ajuda a \textbf{focar} na atividade que estou realizando.\\ \\ 
    \end{tabular}
} \\ \hline
\end{tabular}
\end{table}
\newpage

A persona Afonso Souza de Queiroz (Tabela \ref{tab:Table_persona2}) é caracterizada também, como um jovem de 19 anos do sexo masculino, que cursa Engenharia de Software na UnB. A persona secundária contempla os alunos que desejam pelo menos aprender e avaliar algum conteúdo através de jogos e que estão cursando a disciplina de IHC (23\% referentes aos que pelo menos desejam avaliar seu conhecimento). Ou seja, o público-alvo (persona primária) não fica de fora por conta do propósito do jogo ser auxiliar o aprendizado de um conteúdo, porém outras características devem ser trabalhadas no jogo para abranger aqueles que desejam ter seu conhecimento avaliado. 

Destes que cursam IHC, a parcela da amostra que acaba sendo refletida nesta persona são aqueles que tem experiência de design apenas em disciplinas acadêmicas (30\% da amostra), o que acaba sendo um caso bem similar à da persona primária.

Ainda outra parcela que a persona secundária representa, está relacionada àqueles que pararam  de usar jogos para aprendizagem, mas que conquistaram seu objetivo de estudo com esses jogos (11\% da amostra). Assim como a persona primária, estes revelaram ter uma frequência moderada no uso desse tipo de jogo. O que também demonstra uma certa afinidade deste público com jogos para aprendizagem como a da persona primária.

Pode-se perceber que a persona secundária representa um grupo com mais preferência por recursos não pessoais (internet, video aulas e livros da disciplina) para tirar dúvidas ao invés de falar com colegas, monitores da disciplina e professores. Isso acaba apontando um tendência maior para o jogo não necessariamente abranger recursos interativos com outros usuários, tendo em vista que já para persona primária, uma de suas experiências que não tem tanta relevância assim é a de interação com outros jogadores.

Em relação aos requisitos de qualidade, a persona secundária reflete os mesmos requisitos de preferência da persona primária (Tabela \ref{tab:req-pont}), o Design Atraente, \textit{Feedback} ao Jogador, Regras Fáceis e Facilidade em Aprender. Outros requisitos que para esta persona também fazem parte dos atributos relevantes, são a Acessibilidade (uso adequado das fontes e cores), um Padrão Consistente no Design e a Facilidade de Jogar. Ou seja, estes são fatores secundários, não tão relevantes para o design de um jogo para aprendizagem quando comparados aos outros requisitos de qualidade. E da mesma forma, os três últimos (o jogo ser o principal meio de aprendizagem e a interação com outros jogadores), por estarem com uma pontuação bem mais baixa em relação aos outros requisitos, foram desconsiderados.

Já as experiências esperadas pelos jogadores são as mesmas da persona primária (Tabela \ref{tab:exp-pont}), a Satisfação em jogar e aprender, Confiança na aprendizagem através do jogo, Percepção da Relevância do Conteúdo, Diversão, Desafios e a Atenção Focada nos Jogos. E novamente os dois últimos foram excluídos por apresentarem uma pontuação muito mais baixa em comparação às outras experiências. 

\section{Persona Suplementar}
\label{sec:p3}

A persona suplementar representa mais um grupo de respondentes que têm algumas poucas diferenças em relação à persona primária e a secundária, mas sendo esta contemplada também contribuirá agregando ao jogo alguns elementos e detalhes que podem atrair outros usuários. A Tabela \ref{tab:Table_persona3} apresenta os atributos desta persona.


\begin{table}[htbp]
\centering
\caption{Persona Suplementar}
\label{tab:Table_persona3}
\small
\begin{tabular}{| m{0.25\textwidth} m{0.65\textwidth}|}
\hline \multicolumn{2}{|c|}{\textbf{Identidade}} \\ \hline
& \\

\begin{center} 
\includegraphics[scale=0.06]{figuras/personas/girl-919048_1920.jpg}
Fonte: Pixabay\tablefootnote{https://pixabay.com/photos/girl-portrait-hairstyle-redhead-919048/}
\end{center} 

&

\textbf{Nome: }  Natália Figueiredo

\textbf{Idade:} 23 anos

\textbf{Ocupação:} Estudante de Engenharia de Software na UnB, Faculdade do Gama.

\\ \hline


\multicolumn{2}{|c|}{\textbf{Descrição}} \\ \hline
\multicolumn{2}{|p{15cm}|}{
    \begin{tabular}[c]{@{}l@{}}\\
        Nunca joguei jogos educacionais, pois não conheci nenhum que tinha o propósito de\\ ensinar o que eu desejava. No caso se eu encontrasse um jogo o qual eu pudesse aprender\\ um conteúdo novo e pudesse revisá-lo quando necessário seria interessante. Estou fazendo\\ a disciplina de IHC e desejaria utilizar um jogo que me ajudasse a aprender o conteúdo,\\ pois não tenho muito conhecimento em relação a elaboração de design de interfaces e o\\ pouco que tenho foi somente a partir da disciplina de IHC.\\
        \\
        Geralmente quando vou estudar ou sanar alguma dúvida que tenho sobre o conteúdo eu\\ pesquiso na internet. Em ocasiões bem específicas eu assisto vídeo aulas e também\\ utilizo do material disponibilizado pelo professor. \\
        \\
        Acho que um jogo para se estudar teria de ter um \textbf{design simples}, uma lógica de jogo e\\ \textbf{regras fáceis} de se lembrar. O jogo não deveria ser muito \textbf{difícil}, sendo que o objetivo\\ com ele é aprender. Talvez \textbf{recompensas} e \textbf{mensagens} evidenciando meu progresso\\ seriam interessantes.\\
        \\
        Ao jogar, espero encontrar \textbf{desafios} não muito difíceis, mas que despertem minha\\ \textbf{atenção}. Seria bom, perceber logo de início que o jogo traz um \textbf{conteúdo relevante} e \\que vou \textbf{conseguir aprendê-lo}. E por fim não seria nada ruim sentir \textbf{prazer} em \\aprender e ainda me \textbf{divertir}.\\
        \\
    \end{tabular}
} \\ \hline
\end{tabular}
\end{table}
\newpage

A persona Natália Figueiredo (Tabela \ref{tab:Table_persona3}) é caracterizada como uma jovem de 23 anos do sexo feminino, que cursa Engenharia de Software na UnB. A persona suplementar reflete os alunos que desejam tanto aprender, como avaliar ou revisar algum conteúdo através de jogos e que não estão cursando a disciplina de IHC (27\% da amostra). Neste caso, são contempladas no projeto do jogos os três tipos de motivações para se usar um jogo para aprendizagem, tomando apenas o cuidado de priorizar aqueles que desejam pelo menos aprender um conteúdo com o jogo, depois aqueles que desejam ao menos avaliar seu conhecimento e por fim aqueles que querem revisar o já aprenderam.

Destes que não cursam IHC, a parcela da amostra que é refletida nesta persona são aqueles que não têm experiência de design (11\% da amostra). O que também está relacionado àqueles que mesmo nunca tendo usado jogos para aprendizagem, têm o interesse em jogar (7\% da amostra). Esta persona contemplaria um público que talvez usasse o jogo para IHC de forma antecipada ao ingresso da disciplina de IHC, por curiosidade ou mesmo desejando estudar o conteúdo previamente.

Percebe-se ainda que a persona suplementar abrange todo tipo disponível de recursos para se tirar dúvidas, sejam eles recursos não pessoais ou pessoais. O que reforça ideia de que não há uma forma específica de abordar este recurso extra do jogo, seja o auxílio ao aluno por meios não pessoais ou pessoais.

Em relação aos requisitos de qualidade, a persona suplementar reflete os mesmos requisitos de preferência da persona primária e secundária, o Design Atraente, Feedback ao Jogador, Regras Fáceis e Facilidade em Aprender. Porém além destes requisitos que tiveram destaque para persona primária e dos de Acessibilidade (uso adequado das fontes e cores), um Padrão Consistente no Design e a Facilidade de Jogar que foram apontados na persona secundária, a persona suplementar abrange o requisito do jogo oferecer Pontos e Recompensas. Já o Ranking e a Narrativa continuam de fora, por conta da baixa pontuação quando comparados aos outros requisitos.

Já as experiências esperadas pelos jogadores são as mesmas da persona primária e secundária, a Satisfação em jogar e aprender, Confiança na aprendizagem através do jogo, Percepção da Relevância do Conteúdo, Diversão, Desafios e a Atenção Focada nos Jogos. E novamente os dois últimos foram excluídos por apresentarem uma pontuação muito mais baixa em comparação às outras experiências. 

É interessante que para todas as personas essas experiências almejadas são as mesmas, até as que não foram consideradas relevantes, mas quando analisadas as demais características em relação aos jogos existem algumas distinções. Isto demonstra que mesmo sendo projetado para desenvolver certas experiências, o jogo para aprendizagem pode desenvolver isso por meios, funcionalidades, requisitos e aspectos, algumas vezes diferentes, dependendo do público.


\section{Anti-persona}
\label{sec:p4}

A anti-persona representa um grupo de características que a equipe de designer não precisa se atentar ao desenvolver a interface de um jogo. A Tabela \ref{tab:Table_persona4} apresenta os atributos da anti-persona.



\begin{table}[htbp]
\centering
\caption{Anti-Persona}
\label{tab:Table_persona4}
\small
\begin{tabular}{| m{0.2\textwidth} m{0.7\textwidth}|}
\hline \multicolumn{2}{|c|}{\textbf{Identidade}} \\ \hline
& \\

\begin{center} 
\includegraphics[scale=0.04]{figuras/personas/man-1209494_1920.jpg} 
Fonte: Pixabay\tablefootnote{https://pixabay.com/photos/biceps-aesthetics-body-fitness-2746490/}
\end{center} 

&

\textbf{Nome: }  Rafael Medeiros

\textbf{Idade:} 28 anos

\textbf{Ocupação:} Estudante de Educação Física na UnB - Darcy Ribeiro.

\\ \hline


\multicolumn{2}{|c|}{\textbf{Descrição Geral}} \\ \hline
\multicolumn{2}{|p{15cm}|}{
    \begin{tabular}[c]{@{}l@{}}\\
        Não tenho costume de usar jogos para aprendizagem, devo ter jogado uma vez ou outra,\\ mas\textbf{não me interessei muito} e rapidamente \textbf{o jogo se tornou monótono} pra mim e\\ isso me frustrou, pois não alcancei meu objetivo de estudo. Prefiro jogos esportivos como\\ futebol e basquete. Em relação ao conhecimento acadêmico, me concentro apenas nas\\ disciplinas do meu curso. \\
    \end{tabular}
}

\\ \hline
\multicolumn{2}{|c|}{\textbf{Aspectos de Qualidade}} \\ \hline

\multicolumn{2}{|p{15cm}|}{
    \begin{tabular}[c]{@{}l@{}}\\
        No geral, sou bem \textbf{competitivo}, ``dou o sangue'' para \textbf{conquistar} um gol e fazer um \\\textbf{ponto} em qualquer jogo. Em jogos digitais eu gosto de ser envolvido com uma \textbf{história} \\bacana e fico bastante empolgado com gráficos bem realistas e sofisticados. Sou mais fã \\de jogos com \textbf{trabalho em equipe} do que jogos individuais. Gosto tanto atividades \\ lúdicas que se desse pra aprender com elas, com certeza \textbf{trocaria qualquer livro} pra \\aprender me divertindo. \\
        \\
    \end{tabular}
} \\ \hline
\end{tabular}
\legend{Fonte: Autor}
\end{table}

Primeiramente a anti-persona Rafael Medeiros (Tabela \ref{tab:Table_persona4}) destaca um grupo de respondentes que não são o foco do design do jogo. Estes deixaram de usar esse tipo de jogo por terem tido algum tipo de experiência negativa com jogos para aprendizagem (51\% da amostra). O motivo deste grupo de respondentes de quantidade significante não ter sido representado por uma persona que refletisse um jogador é porque não seria certeza que estes antigos jogadores usariam novamente um jogo para aprendizagem devido a experiência ruim que tiveram. Pode até ser que eles venham usar um jogo seguindo as diretrizes das personas modeladas, porém não se faz necessário se guiar por este grupo representado agora pela anti-persona.

Outros pontos para se destacarem na anti-persona são os requisito de ranking e narrativa, que não são relevantes para um projeto de design, tal como as experiências de interação com outros usuários e o jogos ser o principal meio de estudo. E por fim, é preciso destacar que a anti-persona reflete usuários com algum tipo de limitação e que não é objetivo do designer atender a esse público no momento. 

%tabela com as pontuações de cada requisito e experiência de cada persona