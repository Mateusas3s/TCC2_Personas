
\chapter{Conclusões}
\label{chap:cons}

O capítulo que conclui este trabalho apresenta uma recapitulação do que foi desenvolvido para se alcançar os objetivos deste trabalho. Em seguida são apresentadas brevemente as personas, resultado final deste trabalho. Por fim, é discorrido sobre as contribuições que deste trabalho, suas limitações e as propostas de trabalhos futuros.


\section{Objetivos Alcançados}
O objetivo deste trabalho foi elaborar um elenco de personas para auxiliar o desenvolvimento de jogos digitais para aprendizagem em Interação Humano-Computador. Para isso foram realizados dois procedimento de pesquisa científica e utilizada uma técnica de modelagem de informações em design.

Com o procedimento de pesquisa bibliográfica foi possível identificar as características dos jogos digitais para aprendizagem referente ao objetivo específico ``Identificar características de jogos digitais para aprendizagem'' (OE01). Além disso, a pesquisa bibliográfica contribuiu para o entendimento sobre a técnica de personas.

O segundo procedimento, o de pesquisa com \textit{survey}, foi o que colaborou para a obtenção dos dados de campo sobre alunos de graduação e pós-graduação na área de Ciência da Computação. O \textit{survey} contou com a participação de 166 alunos, que responderam a um questionário digital. Sobre a generalização dos resultados, estes dizem respeito apenas à amostra analisada.

Com o \textit{suvey} foi possível ``definir o perfil de usuários para jogos de aprendizagem em Interação Humano-Computador'', segundo objetivo específico deste trabalho (OE02). Estes dados serviram de insumo para se ``elaborar personas que representem usuários de jogos para aprendizagem em Interação Humano-Computado'', terceiro objetivo específico (OE03). 

Seguindo o método de modelagem de personas defino por \citeonline{cooper07}, foi elaborado um elenco com quatro personas. Este elenco foi modelado de forma a priorizar certos perfis de jogadores e características mais significativas, analisados na amostra coletada no \textit{survey}.

\section{Personas Elaboradas}

Seguindo uma ordem de prioridade foram definidas as personas Victor Matheus Farias (persona primária), seguida de Afonso Souza de Queiroz (persona secundária) e Natália Figueiredo (também persona secundária). Além destas três personas faz parte do elenco a anti-persona, que contempla as características que não devem ser levadas em conta no design de um jogo para aprendizagem em IHC.

A persona primária reflete jogadores-alvo dentro da amostra que são caracterizados por estarem motivados a usarem jogos pelo menos para aprenderem um conteúdo novo sobre IHC. Estes cursam a disciplina de IHC e têm o hábito de usarem moderadamente jogos para aprendizagem. Não têm uma experiência de aprendizado formal quanto ao conteúdo de IHC e geralmente buscam estudar e sanar suas dúvidas sobre conteúdos na internet, porém o auxílio de colegas e professores não é desconsiderado por eles.

Destaca-se ainda nesta persona a importância considerada sobre algumas características de jogos para aprendizagem, como um design atraente, o jogo oferecer \textit{feedbacks} ao jogador, as regras serem fáceis de entender e o jogo ser fácil de aprender. Além destas, é esperado pela persona primária que o jogo tenha características que promovam certas experiências ao jogar. São elas a satisfação em aprender jogando, a confiança na aprendizagem através do jogo, a percepção da relevância do conteúdo apresentado no jogo, diversão, desafios e a o sensação de atenção focada no jogo.

A persona secundária reflete um grupo de jogadores dentro da amostra que são caracterizados por estarem motivados a usarem jogos para aprenderem um conteúdo novo sobre IHC e além disso, avaliar o que aprenderam. Estes também cursam a disciplina de IHC e tem o hábito de usarem moderadamente jogos para aprendizagem assim como a persona primária. Têm experiência em IHC apenas em outras disciplinas e têm o hábito de estudar e sanar suas dúvidas da mesma forma que a persona primária.

Nesta persona destaca-se a importância sobre as mesmas características de jogos para aprendizagem, que foram contempladas na persona primária, porém com algumas a mais, que são aspectos de acessibilidade (uso adequado das fontes e cores no jogo), um padrão consistente no design e a facilidade de se jogar. Além disso, é esperado pela persona secundária as mesmas experiências que as da persona primária.

Já a segunda persona secundária reflete um grupo de jogadores dentro da amostra que são caracterizados por estarem motivados a usarem jogos tanto para aprenderem um conteúdo novo sobre IHC, para avaliar o que aprenderam, quanto para revisarem algo. Estes não cursam a disciplina de IHC e nem têm experiência em design de interfaces, sendo assim estes usariam o jogo para aprenderem antecipadamente o conteúdo de IHC. Assim como as outras personas, têm o hábito de estudar e sanar suas dúvidas tanto de forma pessoal quanto não pessoal.

Nesta persona destaca-se a importância sobre as mesmas características de jogos para aprendizagem em comum entre a persona primária e secundária, que são o jogo ter um design atraente, oferecer \textit{feedback} ao jogador, as regras serem fáceis de entender e o jogo ser fácil de aprender. Porém além destes destaca-se a característica do jogo oferecer pontos e recompensas. Por fim, também é esperado por esta persona as mesmas experiências que as demais.

No caso da anti-persona, ela contempla características que não devem ser consideradas na elaboração de um jogo para aprendizagem em IHC. Desta forma, o jogo não tenta suprir jogadores que de alguma forma se frustraram com jogos para aprendizagem e além disso, destaca que um jogo para aprendizagem em IHC não precisa contemplar um \textit{ranking}, uma narrativa e nem precisa se preocupar em desenvolver meios de interação entre os jogadores, tal como não deve ser o intuito do jogo ser o principal meio de aprendizagem do aluno.

Com este elenco de personas os designer e desenvolvedores tem em mãos de forma priorizada e organizada, os dados sobre o público-alvo para um projeto de jogo. O trabalho traz esta contribuição e algumas outras como é citado na seção seguinte.

\section{Contribuição dos Resultados}

Como o elenco de personas serve para o desenvolvimento de jogos para aprendizagem em IHC, isto contribui para o desenvolvimento e avanço de trabalhos científicos neste tema. Isto já vem acontecendo em outros dois Trabalhos de Conclusão de Curso (ainda não publicados) e com artigos científicos (\cite{deSales_SousaeSilva_2020, silva_sales_mendes2021, sales_joao_2021}) do projeto ``Recursos Digitais Didáticos para Interação Humano-Computado''. E além da contribuição dentro do tema jogos para aprendizagem em IHC, esta pesquisa pode inspirar outros pesquisadores que desejarem replicá-la em outras áreas de conhecimento.

Dentro do ciclo de vida da produção de um jogo, assim como é num projeto de design de interfaces de sistemas digitais, estas personas podem fazer parte do processo. Por exemplo, no caso do \textit{Playcentric Design Process} \cite{Fullerton_2008}, as personas pode ajudar nas atividades de concepção do jogo, definindo as metas de experiência do jogo a partir das experiências desejadas das próprias personas.

As personas ainda podem ser utilizadas em atividades de teste e validação de elementos a serem incorporados no jogo. E também podem servir como norte para a equipe de desenvolvimento montar seu \textit{product backlog}, definir os requisitos não funcionais do sistema, prototipagem, tal como priorizá-los e assim ter sempre em mente quais critérios devem ser atendidos.

\section{Limitações do Trabalho}

Como foi apontado, a análise diz respeito apenas a amostra coletada e isto é uma limitação deste trabalho. E até mesmo, caso seja projetado um jogo a partir destas personas pode ser que o mesmo não venha contemplar o perfil dos alunos que estiverem cursando a disciplina de IHC, pois existe a possibilidade das características aqui definidas para as personas não serem as mesmas dos alunos de IHC que usarem o jogo.

Outro ponto é que as personas foram modeladas a partir de apenas um papel de usuário, que são os alunos de ensino superior na área de Ciência da Computação. Porém outros papéis estão envoltos no processo de ensino e aprendizagem, como professores e monitores de disciplina.

Outra limitação a ser destacada é que com apenas as metas de experiência e os requisitos de qualidade (requisitos não funcionais) ainda fica ``nebuloso'' definir quais os mecanismos e funcionalidades o jogo deve ter. São estes elementos (requisitos funcionais) que vão definir o estilo jogo. Por exemplo, se o jogo vai ser de perguntas e respostas, um RPG (Role-Playing Game), jogo de estratégia etc. 

Além destas, nem todas as etapas deste trabalho foram realizadas com uma equipe de pesquisadores-desenvolvedores (alunos participantes do projeto), mas sim de forma individual, salvo orientação dos coordenadores do projeto. Cada aluno teve um trabalho próprio sendo desenvolvido no projeto de pesquisa  ´´Recursos Digitais Didáticos para Interação Humano-Computador''. Sendo assim, algumas etapas deste trabalho poderiam ter uma qualidade maior caso fossem realizadas com uma equipe de desenvolvedores.

\section{Sugestões de Trabalhos Futuros}

Primeiramente, como sugestão de trabalhos futuros, seria importante coletar uma amostra mais relevante, na qual fosse possível expandir o nível da generalização. Com isto seria possível ter um elenco de personas mais relevante para refletir o público-alvo de um projeto de jogo para aprendizagem. Além disto seria interessante o projeto das personas agregar mais papéis de usuários, além dos alunos. Caso fossem contemplados professores e monitores de disciplina, o design do jogo poderia ser mais relevante do que simplesmente um jogo atendendo os alunos. 

Outra sugestão é que esta pesquisa poderia ser aplicada em outras áreas de conhecimento e assim modelar personas com base em características de jogos específicas para tal área. Isto se deve, pois a base da literatura usada para levantar as características de jogos neste trabalho foram os trabalhos de \citeonline{deSales_SousaeSilva_2020}, com características de jogos em IHC, e o trabalho de \citeonline{Petri_Wangenheim_2019}, com características de jogos na computação.

Além da ampliação da generalização da amostra e aplicação desta pesquisa em outras áreas de conhecimento, seria interessante entender quais gênero de jogos e quais funcionalidades (requisitos funcionais) e mecânicas de jogos seriam os mais relevantes para o público-alvo. Uma sugestão seria expandir o \textit{survey} para coletar também estes dados e realizar uma revisão na literatura tendo este objetivo.

Um outro trabalho futuro sugerido seria o desenvolvimento de um jogo usando estas personas. Com isso seria possível comparar as preferências entre os usuários do jogo desenvolvido e o perfil dos jogadores representados pelas personas que auxiliaram o desenvolvimento do jogo. Além disso, outro trabalho poderia ser a verificação se o jogo atendeu as expectativas dos jogadores, atendendo os requisitos e metas experiência definidos. 

Por fim, para que este processo ocorra de forma mais ágil, a coleta, a elaboração do elenco das personas e o desenvolvimento do jogo, deve contar com uma equipe de designers e desenvolvedores. Estes devem estar familiarizados com técnicas de pesquisa, levantamento e análise de requisitos, como a de Revisão Sistemática de Literatura, \textit{Survey} e o método de elaboração de personas, além de habilidades como programação e design de interfaces.
