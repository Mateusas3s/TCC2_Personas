
\chapter{Conclusões}
\label{chap:cons}

O objetivo deste trabalho foi elaborar um elenco de personas para auxiliar o desenvolvimento de jogos digitais para aprendizagem em Interação Humano-Computador. Para isso foram realizados dois procedimento de pesquisa científica e utilizada uma técnica de modelagem de informações em design. 

Com o procedimento de pesquisa bibliográfica foi possível identificar as características dos jogos digitais para aprendizagem e as características do seu processo de desenvolvimento, referente ao objetivo específico OE01. Além deste objetivo específico, a pesquisa bibliográfica contribuiu para o entendimento sobre a técnica de personas.

O segundo procedimento, o de pesquisa com \textit{survey}, foi o que colaborou para a obtenção de dados de campo sobre um público-alvo de alunos de graduação e pós-graduação na área de Ciência da Computação. O \textit{survey} contou com a participação de 166 alunos, que responderam a um questionário digital. Sobre a generalização dos resultados, estes dizem respeito apenas à amostra analisada.

Estes dados serviram de insumo na elaboração de um elenco de personas que representasse este público-alvo quanto ao uso de jogos para aprendizagem em Interação Humano-Computador, objetivo específico OE02. Cada persona elencada abrange características significativas que foram analisadas na amostra coletada no \textit{survey}.

O elenco das personas foi modelado de forma a priorizar certos perfis de jogadores e características mais significativas, analisados dentro da própria amostra. Sendo que as mais relevantes foram personificadas na figura de Victor Matheus Farias (persona primária), seguida de Afonso Souza de Queiroz (persona secundária) e Natália Figueiredo (persona suplementar). Além destas três personas faz parte do elenco a anti-persona, que contempla as características que não devem ser levadas em conta no design de um jogo para aprendizagem em IHC.

A persona primária reflete jogadores-alvo dentro da amostra que são caracterizados por estarem motivados a usarem jogos para pelo menos aprenderem um conteúdo novo sobre IHC. Estes cursam a disciplina de IHC e tem o hábito de usarem moderadamente jogos para aprendizagem. Não têm uma experiência de aprendizado formal quanto ao conteúdo de IHC e geralmente buscam estudar e sanar suas dúvidas sobre conteúdos na internet, porém o auxílio de colegas e professores não é desconsiderado.

Destaca-se ainda nesta persona a importância considerada sobre algumas características de jogos para aprendizagem, como um design atraente, o jogo oferecer \textit{feedback} ao jogador, as regras serem fáceis de entender e o jogo ser fácil de aprender. Além destas, é esperado pela persona primária que o jogo tenha características que promovam certas experiências ao jogar. São elas a satisfação em aprender jogando, a confiança na aprendizagem através do jogo, a percepção da relevância do conteúdo apresentado no jogo, diversão, desafios e a o sensação de atenção focada no jogo.

A persona secundária reflete um grupo de jogadores dentro da amostra que são caracterizados por estarem motivados a usarem jogos para além de aprenderem um conteúdo novo sobre IHC, também avaliar o que aprenderam. Estes também cursam a disciplina de IHC e tem o hábito de usarem moderadamente jogos para aprendizagem assim como a persona primária. Têm experiência em IHC apenas em outras disciplinas e têm o hábito de estudar e sanar suas dúvidas da mesma forma que a persona primária.

Nesta persona destaca-se a importância sobre as mesmas características de jogos para aprendizagem, que foram contempladas na persona primária porém com algumas a mais, que são aspectos de acessibilidade (uso adequado das fontes e cores no jogo), um padrão consistente no design e a facilidade de se jogar. Além destas, é esperado pela persona secundária as mesmas experiências que as da persona primária.

Já persona suplementar reflete um grupo de jogadores dentro da amostra que são caracterizados por estarem motivados a usarem jogos tanto para aprenderem um conteúdo novo sobre IHC, para avaliar o que aprenderam, quanto para revisarem algo. Estes não cursam a disciplina de IHC e usariam um jogo deste tipo para aprenderem de forma antecipada o conteúdo de IHC. Estes não têm experiência em IHC e assim como as outras personas, têm o hábito de estudar e sanar suas dúvidas da mesma forma pessoal e não pessoal.

Nesta persona destaca-se a importância sobre as mesmas características de jogos para aprendizagem em comum entre a persona primária e secundária, que são o jogo ter um design atraente, oferecer \textit{feedback} ao jogador, as regras serem fáceis de entender e o jogo ser fácil de aprender. Porém além destes destaca-se a característica do jogo oferecer pontos e recompensas. Por fim, também é esperado por esta persona as mesmas experiências que as demais.

No caso da anti-persona, ela contempla características que não devem ser consideradas na elaboração de um jogo para aprendizagem em IHC. No caso, o jogo não tenta suprir jogadores que de alguma forma se frustraram anteriormente com jogos para aprendizagem, assim como destaca que um jogo para aprendizagem em IHC voltado para o público-alvo desta pesquisa não precisa contemplar um \textit{ranking} e narrativa, e nem precisa se preocupar em desenvolver meios de interação entre os jogadores, tal como não deve ser o intuito do jogo ser o principal meio de aprendizagem do aluno.

Como foi comentado, a análise diz respeito apenas a amostra coletada e isto é uma limitação deste trabalho. E até mesmo, caso seja projetado um jogo a partir destas personas pode ser que o mesmo não venha contemplar o perfil dos alunos que estiverem cursando a disciplina de IHC, pois existe a possibilidade das características aqui definidas para as personas não serem as mesmas dos alunos de IHC que usarem o jogo.

No entanto este trabalho serve de base para futuros estudos, os quais podem coletar uma amostra maior de participantes e assim generalizar os resultados, ou fazer a coleta com os alunos um semestre anterior à disciplina de IHC, desta forma as chances das personas projetadas refletirem os alunos reais que estiverem cursando IHC seria maior.

Outro ponto é que, para que este processo ocorra de forma mais ágil, a coleta, elaboração do elenco das personas e o desenvolvimento do jogo, deve contar com uma equipe de designers e desenvolvedores com familiaridade com as técnicas de pesquisa com \textit{survey} e da elaboração de personas, assim como habilidades de desenvolvimento de software e prática de metodologias ágeis.

Sendo assim, o elenco de personas aqui desenvolvido serve para o desenvolvimento de jogos para aprendizagem em IHC. Isto contribui para o desenvolvimento e avanço de trabalhos dentro do projeto de pesquisa Recursos Digitais Didáticos para Interação Humano-Computado desenvolvida na UnB, Faculdade do Gama, quanto para outros pesquisadores que desejarem replicar a pesquisa em outras áreas de conhecimento.

Dentro do ciclo de vida da produção de um jogo, assim como é num projeto de design de interfaces de sistemas digitais, estas personas podem fazer parte do processo. Por exemplo, no caso do \textit{Playcentric Design Process} \cite{Fullerton_2008}, as personas pode ajudar nas atividades de concepção do jogo, definindo as metas de experiência do jogo a partir das experiências desejadas das próprias personas.

As personas ainda podem ser utilizadas em atividades de teste e validação de elementos a serem incorporados no jogo. E também podem servir como norte para a equipe de desenvolvimento montar seu \textit{product backlog}, definir os requisitos não funcionais do sistema, tal como priorizá-los e assim ter sempre em mente quais critérios devem ser atendidos.

Foi percebido que é esperado pelos jogadores, que jogos para aprendizagem tenham aspectos que façam com que eles alcancem seus objetivos de estudo, porém dentre estes aspectos existem aqueles que não estão ligados ao aprendizado em si (diversão, desafio, pontos e recompensas), mas que contribuem para isto. Como trabalhos futuros, além da ampliação da amostra e aplicação desta pesquisa em outras áreas de conhecimento, seria interessante entender quais as preferências de tipos de jogos dos alunos e quais funcionalidades (requisitos funcionais) podem contribuir para gerar as experiências desejadas. 

Por fim, um trabalho futuro sugerido seria o desenvolvimento de um jogo usando estas personas. Com isso seria possível comparar as percepções dos jogadores com as personas que auxiliaram o desenvolvimento do jogo. Além disso poderia ser verificado se o jogo atendeu as expectativas dos jogadores, atendendo os requisitos e metas experiência definidos. 
