\chapter{Introdução}
\label{chap:intro}

Neste capítulo é apresentado o contexto que envolve o tema deste trabalho, assim como sua motivação e justificativa. Além disto, estão apresentados os objetivos geral e específicos, e a estrutura deste documento. 

As tecnologias digitais conquistam mais espaço e se consolidam no cotidiano das pessoas \cite{Sales2020}. Para que ocorram interações de qualidade entre seres humanos e essas tecnologias, as áreas de Engenharia de Software (ES) e de Interação Humano-Computador (IHC) buscam desenvolver conhecimento e mecanismos para auxiliar na criação de interfaces, as quais servem para que o usuário possa interagir com tais tecnologias digitais \cite[p. 3, 8-10]{BarbosaEtAl2021}.  

\citeonline{seffah2005} investigaram a integração entre técnicas e métodos nas áreas de Engenharia de Software e Interação Humano-Computador. Mesmo tendo abrangências diferentes, existem pontos de interseção entre ES e IHC. Essas interseções estão relacionados às características de um processo de desenvolvimento que se preocupa com a qualidade em uso do sistema interativo e isto é incorporado às atividades, técnicas, métodos e processos \cite[p. 114-115]{BarbosaEtAl2021}.

O processo de design de interface envolve basicamente três atividades: análise do problema, síntese de uma intervenção (ou solução) e avaliação desta intervenção \cite{lawson2006}. Estas são modeladas de formas diferentes dependendo do enfoque e detalhamento que seja necessário ao projeto \cite[p. 98-99]{BarbosaEtAl2021}. A atividade de análise envolve a pesquisa das necessidades do usuário, suas características e expectativas, a fim de ser elaborada a proposta de intervenção \cite[p. 121]{BarbosaEtAl2021}

É importante no processo de design de interface haver a organização e modelagem dos dados coletados sobre os usuários e demais elementos envolvidos no problema. Os modelos formados nesta atividade de análise são um recorte do mundo explorado na pesquisa e que representam o entendimento do designer sobre o problema \cite[p. 151]{BarbosaEtAl2021}. 

As representações desta modelagem podem ser o perfil de usuário, \textit{persona}, cenários e modelos de tarefas. Estes podem ser apresentados sob diferentes perspectivas, com um determinado foco e níveis diferentes de detalhes, o que vai depender das necessidades no projeto \cite[p. 151]{BarbosaEtAl2021}.

A técnica de \textit{personas} por exemplo, aborda uma estratégia de modelagem, que visa projetar o design de um produto para indivíduos bem específicos e com necessidades específicas ao invés de projetar algo que tente abranger vários usuários distintos \cite[p. 77]{cooper07}. Essa técnica é uma ferramenta poderosa para a elaboração do design, pois uma vez que ela atenda os objetivos das personas elencadas e que as personas tenham sido bem elaboradas, o design da interface satisfará os usuários reais \cite[p. 77]{cooper07}.

Em um cenário no qual o público-alvo do projeto de design envolve várias pessoas e que a presença frequente do usuário nas etapas do projeto seja difícil, o uso das personas pode auxiliar a equipe de desenvolvimento. No caso, as personas podem ser utilizadas em algumas etapas de um projeto de design, suprindo a participação do usuário em atividades de concepção, ideação, validação e tomada de decisão, por exemplo \cite[p. 80]{Vianna_2014} \cite[p. 155]{BarbosaEtAl2021}.

Como citado anteriormente, IHC aliada à ES têm uma preocupação com o desenvolvimento de interfaces centradas no usuário. Os cursos de graduação e pós-graduação na área de Ciência da Computação têm objetivo de formar profissionais qualificados para desenvolver sistemas digitais com qualidade, levando em conta as necessidades e expectativas do usuário \cite[p. 89,  90]{acm_curricula} \cite[p. 8-16]{BarbosaEtAl2021}. No entanto, a visão de IHC centrada no usuário pode ser tratada como secundária pelos profissionais que desenvolvem software. Isto se deve a falta de compreensão desses profissionais, que focam mais na parte interna dos sistemas, como código de algoritmos e banco de dados \cite{sommariva}.

A academia tem investido em novas abordagens e tecnologias como recursos auxiliares no processo de ensino e aprendizagem, apoiando assim, o desenvolvimento de atividades pedagógicas inovadoras e colaborativas \cite{battistella, brito, Sales2020, Sales2020UsoTDS}. Os jogos sérios fazem parte dessas abordagens.

Esse tipo de jogo pode ser aplicado em diversas áreas, tais como saúde, publicidade, política e educação. Jogos sérios também incluem jogos para aprendizagem \cite{Becker_2021}. Estes vêm se tornando cada vez mais populares na educação em computação, pois podem aumentar a eficácia e o engajamento da aprendizagem \cite{battistella, brito, queiroz, sales_climaco2016}.

Esse contexto incentivou a criação do projeto de pesquisa ``Recursos Digitais Didáticos para Interação Humano-Computador\footnote{\url{https://github.com/RecursosDigitaisdeEnsinoAprendizagemIHC}}'' (RDDIHC), que visa o desenvolvimento de recursos auxiliares ao processo de ensino e aprendizagem em IHC. Uma das frentes de pesquisa deste projeto é a de jogos para aprendizagem em IHC voltados para cursos de graduação e pós-graduação na área de Ciência da Computação, do qual este trabalho de conclusão de curso faz parte.

Estes fatores motivaram o desenvolvimento do presente trabalho, que objetiva fazer uso da técnica de personas para auxiliar no desenvolvimento de jogos digitais para aprendizagem em IHC. Outro ponto, mencionado anteriormente, que impulsionou este trabalho de conclusão de curso foi o cenário atual de trabalho remoto. 

Este ambiente abre margem para o uso da técnica em projetos de natureza remota, justificando assim a escolha da técnica de personas como objeto de estudo e ferramenta auxiliar para apoiar os trabalhos do projeto RDDIHC. Nas duas seções a seguir são apresentados com mais detalhes os objetivos geral e específicos.


\section{Objetivo Geral}
\label{sec:objetivos}

O objetivo geral deste trabalho é elaborar um elenco de personas para auxiliar o desenvolvimento de jogos digitais para aprendizagem em Interação Humano-Computador.

\section{Objetivo Específico}
\label{ssec:obj_especifico}
\begin{itemize}

    
     \item \textbf{OE01} - Identificar características dos jogos digitais para aprendizagem e de seu desenvolvimento;
     % Revisão Bibliográfica quanto às características de jogos na área de IHC;
     
     \item \textbf{OE02} - Elaborar um elenco de personas para jogos para aprendizagem em Interação Humano-Computador para alunos de graduação e pós-graduação na área de Ciência da Computação;
    % Survey
    % metodologia do "Face 3" para construção de personas
    
    % objetivo a partir da expansão dos dados
    %\item \textbf{OE04} - Mapear atividades dentro do desenvolvimento de jogos, onde podem ser usadas as personas;
     % Revisão Bibliográfica quanto ao desenvolvimento de jogos 
    % mapear atividades dentro de cada etapa do processo de desenvolvimento do jogo para  aplicar a técnica de personas
    % coletar feedback dos desenvolvedores dos jogos sobre os benefícios do uso das personsa no projeto.
    
\end{itemize}

%\section{Planejamento do Trabalho}

% fazer tipo uma linha do tempo com o que aconteceu até se originar a ideia do projeto original - > pode ser que fique melhor tudo na metodologia

\section{Estrutura do Trabalho}

Este trabalho está estruturado em cinco capítulos além da introdução. O Capítulo \ref{chap:ref} apresenta os principais conceitos que envolvem este trabalho e como eles se relacionam. Neste capítulo está descrito uma base sobre IHC, Personas e Jogos.

O Capítulo \ref{chap:Metodo} apresenta a estrutura do processo metodológico utilizado neste trabalho. Nele é relatado a metodologia usada na pesquisa científica e da modelagem de design, com a construção das personas. 

Nos Capítulos \ref{chap:result} e \ref{chap:ele-pers} são apresentados os resultados e a discussão sobre a análise feita. O Capítulo \ref{chap:result} apresenta as características da amostra. Já o Capítulo \ref{chap:ele-pers} relata as personas modeladas para o design de um jogo em IHC. 

Por fim, são apresentadas as ideias e pontos conclusivos no Capítulo \ref{chap:cons}. 

%O capítulo Apresentação dos Resultados...

%O capítulo Considerações Finais ...