%\chapter{Desenvolvimento do Jogo}
\chapter{Resultados}
\label{chap:result}

Neste capítulo são apresentados os resultados...

% apresentar um pouco da análise dos dados do survey beeeem resumido e caracterizar a população

Na pesquisa com \textit{survey} obteve-se 166 registros de respostas válidas. Observa-se que a idade dos respondentes varia entre 18 e 53 anos, sendo que 75\% deles possuem até 23 anos de idade, 128 respondentes (77\%) são do sexo masculino, 36 (22\%) do sexo feminino e dois dos respondentes preferiram não responder esta pergunta. Dentre as instituições de ensino que participaram da pesquisa, foram obtidas 126 respostas de alunos ou ex-alunos da UnB, 24 da UFMS, 10 da UFAM, 4 da UFMT, 1 resposta da UCSAL e outra da UTFPR.


% Em relação aos respondentes e o curso de Interação Humano-Computador, foi observado que 44 respondentes não haviam feito a disciplina, 26,5\% do total; 57 já haviam feito a disciplina, 34,3\% do total; e 65 estavam fazendo a disciplina durante esta pesquisa, 39,2\% do total. A Figura \ref{Fig:curso.png}, da relação dos respondentes e o curso de IHC.

% \begin{figure}[htbp]
% 	\centering
% 	\caption{Relação do respondente com o curso de IHC}
% 	\includegraphics[keepaspectratio=true,scale=0.4]{figuras/apendice/graficos_survey/curso.png}
% 	\label{Fig:curso.png}
% \end{figure}


% falar sobre a pergunta se os alunos conheciam a técnbica de personas


Na etapa de coleta de dados foi elaborado um \textit{survey}, no qual foi desenvolvido e aplicado um questionário de pesquisa, que teve como objetivo identificar o público-alvo e elicitar mais alguns requisitos. O relatório do \textit{survey} é descrito no Apêndice \ref{ap:questionario} assim como o próprio questionário. Para finalizar essa etapa e fechar o escopo foram elaboradas personas, estas guiaram a definição, especificação e validação dos requisitos identificados.

% ligar paragrafos


O elenco de personas criado conta com a persona primária, Victor Matheus Farias; como persona secundária, Afonso de Souza Queiroz; como persona suplementar, Natália Figueiredo; e a anti-persona Rafael Medeiros.

\newpage

\begin{table}[htbp]
\centering
\caption{Persona Primária}
\label{tab:Table_persona1}
\small
\begin{tabular}{| m{0.25\textwidth} m{0.65\textwidth}|}
\hline \multicolumn{2}{|c|}{\textbf{Identidade}} \\ \hline
& \\

\begin{center} 
\includegraphics[scale=0.06]{figuras/personas/portrait-3353699_1920.jpg} 
Fonte: Pixabay\tablefootnote{https://pixabay.com/photos/portrait-people-adult-man-face-3353699/}
\end{center} 



&

\textbf{Nome: } Victor Matheus Farias

\textbf{Idade:} 19 anos

\textbf{Ocupação:} Estudante de Engenharia de Software na UnB, Faculdade do Gama.

\\ \hline


\multicolumn{2}{|c|}{\textbf{Descrição}} \\ \hline
\multicolumn{2}{|p{15cm}|}{
    \begin{tabular}[c]{@{}l@{}}\\
    Aprender um conteúdo novo é meu o principal objetivo ao usar jogos educacionais.\\ Atualmente eu uso alguns, mas com uma frequência moderada, não gosto de gastar\\ muito tempo com jogos. Vejo-os como ferramentas que me auxiliam no processo de \\aprendizagem. Estou cursando a disciplina de IHC e não tenho muito conhecimento \\em relação a elaboração de design de interfaces e o pouco que tenho está somente no \\âmbito disciplinar. Desejaria utilizar um jogo que me ajudasse a aprender o conteúdo.\\ 
    \\
    Geralmente quando vou estudar ou sanar alguma dúvida que tenho sobre o conteúdo \\eu pesquiso na internet, em alguns casos assisto vídeo aulas e também utilizo do \\material disponibilizado pelo professor.\\
    \\
    Um jogo educacional pra mim, deve \textbf{responder bem às minhas ações}, me ensinando\\ caso eu erre por exemplo; deve ter um \textbf{design legal}, talvez com um tema; deve ser \\\textbf{simples de se aprender a jogar}, não tendo regras extensas e tutorias longos; e \\também deve ser \textbf{fácil de jogar}. \\
    \\
    E por fim o que eu espero de um jogo educacional é sentir \textbf{satisfação}, aprender com um\\ pouco de \textbf{desafio} e \textbf{diversão} é muito bom. Ter a percepção da \textbf{relevância} do assunto\\ que estou aprendendo é algo que me dá \textbf{confiança} que vou atingir meu objetivo de\\ estudo. E mesmo não gostando de gastar muito tempo com jogos, mas se eu sentir que \\estou sendo produtivo com certeza vou me manter \textbf{focado}.\\ \\
    \end{tabular}
} \\ \hline
\end{tabular}
\end{table}
\newpage
\begin{table}[htbp]
\centering
\caption{Persona Secundária}
\label{tab:Table_persona2}
\small
\begin{tabular}{| m{0.25\textwidth} m{0.65\textwidth}|}
\hline \multicolumn{2}{|c|}{\textbf{Identidade}} \\ \hline
& \\

\begin{center} 
\includegraphics[scale=0.06]{figuras/personas/model-2911332_1920.jpg} 
Fonte: Pixabay\tablefootnote{https://pixabay.com/photos/model-businessman-corporate-2911332/}
\end{center} 

&

\textbf{Nome: } Afonso Souza de Queiroz

\textbf{Idade:} 19 anos

\textbf{Ocupação:} Estudante de Engenharia de Software na UnB, Faculdade do Gama

\\ \hline


\multicolumn{2}{|c|}{\textbf{Descrição}} \\ \hline
\multicolumn{2}{|p{15cm}|}{
     \begin{tabular}[c]{@{}l@{}}\\
        Já joguei alguns jogos educacionais onde meu principal objetivo era aprender um\\ conteúdo novo. Além disso era interessante quando o jogo possibilitava que eu revisasse\\ o conteúdo e avaliasse o que tinha aprendido. Eu jogava com certa moderação. Por mais\\ que atualmente eu não esteja usando algum jogo educacional, eu parei de jogar pois \\alcancei meus objetivos de estudo e me sentiria da mesma forma se houvesse algum\\ jogo que me ajudasse nas matérias da faculdade.\\ 
        \\
        Ainda não fiz a disciplina de IHC, por isso não tenho muito conhecimento em relação\\ a elaboração de design de interfaces, sendo que o pouco que tenho veio de projetos de\\ outras disciplinas e atividades extra curriculares.\\
        \\
        Gosto mais de estudar sozinho, pesquisando na internet e indo por conta própria atrás\\ de materiais de estudo. Apenas quando não acho nada relevante, recorro aos materiais\\ disponibilizados pelo professor.\\
        \\
        Para mim, um jogo onde vou aprender deve ter um design com um \textbf{padrão simples},\\ mas que desperte \textbf{interesse}; deve me dar bons \textbf{feedbacks} a cada interação que faço\\ no jogo; deve ser algo \textbf{simples de aprender como jogar} e \textbf{simples de jogar}; e caso\\ tenha textos sobre o conteúdo, seria bom que os textos fossem bem objetivos e com\\ uma cor e fonte que facilitasse a leitura.\\ 
        \\
        Espero ter a sensação logo de cara que \textbf{não vou perder meu tempo} com o jogo, mas\\ que ao final vou perceber o quanto foi \textbf{importante aprender o conteúdo} e o quanto\\ foi \textbf{satisfatório} ter me \textbf{divertido} aprendendo. E espero também de certa forma ser\\ \textbf{desafiado}, pois isso me ajuda a \textbf{focar} na atividade que estou realizando.\\ \\ 
    \end{tabular}
} \\ \hline
\end{tabular}
\end{table}
\newpage

\begin{table}[htbp]
\centering
\caption{Persona Suplementar}
\label{tab:Table_persona3}
\small
\begin{tabular}{| m{0.25\textwidth} m{0.65\textwidth}|}
\hline \multicolumn{2}{|c|}{\textbf{Identidade}} \\ \hline
& \\

\begin{center} 
\includegraphics[scale=0.06]{figuras/personas/girl-919048_1920.jpg}
Fonte: Pixabay\tablefootnote{https://pixabay.com/photos/girl-portrait-hairstyle-redhead-919048/}
\end{center} 

&

\textbf{Nome: }  Natália Figueiredo

\textbf{Idade:} 23 anos

\textbf{Ocupação:} Estudante de Engenharia de Software na UnB, Faculdade do Gama.

\\ \hline


\multicolumn{2}{|c|}{\textbf{Descrição}} \\ \hline
\multicolumn{2}{|p{15cm}|}{
    \begin{tabular}[c]{@{}l@{}}\\
        Nunca joguei jogos educacionais, pois não conheci nenhum que tinha o propósito de\\ ensinar o que eu desejava. No caso se eu encontrasse um jogo o qual eu pudesse aprender\\ um conteúdo novo e pudesse revisá-lo quando necessário seria interessante. Estou fazendo\\ a disciplina de IHC e desejaria utilizar um jogo que me ajudasse a aprender o conteúdo,\\ pois não tenho muito conhecimento em relação a elaboração de design de interfaces e o\\ pouco que tenho foi somente a partir da disciplina de IHC.\\
        \\
        Geralmente quando vou estudar ou sanar alguma dúvida que tenho sobre o conteúdo eu\\ pesquiso na internet. Em ocasiões bem específicas eu assisto vídeo aulas e também\\ utilizo do material disponibilizado pelo professor. \\
        \\
        Acho que um jogo para se estudar teria de ter um \textbf{design simples}, uma lógica de jogo e\\ \textbf{regras fáceis} de se lembrar. O jogo não deveria ser muito \textbf{difícil}, sendo que o objetivo\\ com ele é aprender. Talvez \textbf{recompensas} e \textbf{mensagens} evidenciando meu progresso\\ seriam interessantes.\\
        \\
        Ao jogar, espero encontrar \textbf{desafios} não muito difíceis, mas que despertem minha\\ \textbf{atenção}. Seria bom, perceber logo de início que o jogo traz um \textbf{conteúdo relevante} e \\que vou \textbf{conseguir aprendê-lo}. E por fim não seria nada ruim sentir \textbf{prazer} em \\aprender e ainda me \textbf{divertir}.\\
        \\
    \end{tabular}
} \\ \hline
\end{tabular}
\end{table}
\newpage


\begin{table}[htbp]
\centering
\caption{Anti-Persona}
\label{tab:Table_persona4}
\small
\begin{tabular}{| m{0.2\textwidth} m{0.7\textwidth}|}
\hline \multicolumn{2}{|c|}{\textbf{Identidade}} \\ \hline
& \\

\begin{center} 
\includegraphics[scale=0.04]{figuras/personas/man-1209494_1920.jpg} 
Fonte: Pixabay\tablefootnote{https://pixabay.com/photos/biceps-aesthetics-body-fitness-2746490/}
\end{center} 

&

\textbf{Nome: }  Rafael Medeiros

\textbf{Idade:} 28 anos

\textbf{Ocupação:} Estudante de Educação Física na UnB - Darcy Ribeiro.

\\ \hline


\multicolumn{2}{|c|}{\textbf{Descrição Geral}} \\ \hline
\multicolumn{2}{|p{15cm}|}{
    \begin{tabular}[c]{@{}l@{}}\\
        Não tenho costume de usar jogos para aprendizagem, devo ter jogado uma vez ou outra,\\ mas\textbf{não me interessei muito} e rapidamente \textbf{o jogo se tornou monótono} pra mim e\\ isso me frustrou, pois não alcancei meu objetivo de estudo. Prefiro jogos esportivos como\\ futebol e basquete. Em relação ao conhecimento acadêmico, me concentro apenas nas\\ disciplinas do meu curso. \\
    \end{tabular}
}

\\ \hline
\multicolumn{2}{|c|}{\textbf{Aspectos de Qualidade}} \\ \hline

\multicolumn{2}{|p{15cm}|}{
    \begin{tabular}[c]{@{}l@{}}\\
        No geral, sou bem \textbf{competitivo}, ``dou o sangue'' para \textbf{conquistar} um gol e fazer um \\\textbf{ponto} em qualquer jogo. Em jogos digitais eu gosto de ser envolvido com uma \textbf{história} \\bacana e fico bastante empolgado com gráficos bem realistas e sofisticados. Sou mais fã \\de jogos com \textbf{trabalho em equipe} do que jogos individuais. Gosto tanto atividades \\ lúdicas que se desse pra aprender com elas, com certeza \textbf{trocaria qualquer livro} pra \\aprender me divertindo. \\
        \\
    \end{tabular}
} \\ \hline
\end{tabular}
\legend{Fonte: Autor}
\end{table}

% contar mais um pouco da historinha de como foram definidas as características de cada uma, mostrando o link  da frequencia das respostas com o porque da característica da persona.  


