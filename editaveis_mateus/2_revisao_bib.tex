\chapter{Referencial Teórico}
\label{chap:ref}

% apresentar o livros e citar artigos


Neste capítulo são apresentados os conceitos base que envolvem este trabalho. São tratadas a teoria relacionada às personas (Seção \ref{sec:personas}) e as características dos jogos para aprendizagem (Seção \ref{sec:jogos-aprend}). %Além disto são apresentados trabalhos correlatos na Seção \ref{sec:trab_cor}. 

% fazer comparação às personas, talvez até na discussão
% - cenários, perfil de usuário, modelagem de tarefas e um outro lá

\section{Personas}
\label{sec:personas}

Nesta seção são apresentados conceitos sobre a técnica de personas. São abordados a definição de personas, suas características, além da composição de um elenco de personas para um projetos.

\subsection{Definição de Persona}

As personas são personagens fictícios, arquétipos hipotéticos de um grupo de usuários reais, criadas para descrever um usuário típico, que são definidas principalmente por seus objetivos \cite{cooper07, pruitt, cooper99}. Embora fictícia, uma \textit{persona} é definida com um rigor de detalhes de forma que represente bem o público real de usuários que farão uso da interface projetada \cite[p. 154]{BarbosaEtAl2021}. Na Figura \ref{Fig:ex-persona.png} é apresentado um exemplo de persona.

\begin{figure}[htbp]
	\centering
	\includegraphics[keepaspectratio=true,scale=0.6]{figuras/metodologia/exemplo-persona.png}
	\caption{Exemplo de uma persona - \citeonline{barbosa_silva}}
	\label{Fig:ex-persona.png}
\end{figure}

Elas são definidas a partir de dados de pesquisa de campo, ou seja, as personas refletem as características de pessoas reais. Elas se caracterizam por atributos demográficos como sexo e faixa etária, atendem por um nome, além de conterem aspectos comportamentais, motivação e objetivos \cite[p. 81]{Vianna_2014}. Como é definido por \citeonline{Courage_Baxter_2005}, citado por \cite[p. 153-154]{BarbosaEtAl2021} uma persona possui as seguintes características: 

\begin{itemize}
    \item \textbf{identidade:} para dar mais realismo à persona, é importante que ela tenha um nome, idade, outros dados demográficos e uma foto;

    \item \textbf{status:} define o tipo das personas. Este atributo reflete a prioridade de uma persona classificando-a em primária, secundária, suplementar, cliente, servida ou anti-persona;

    \item \textbf{objetivos:} condição final a ser atingida, o alvo da persona. Os objetivos de uma persona não se restringem apenas ao produto, podendo ser por exemplo um objetivo pessoal, corporativo, técnico e prático;

    \item \textbf{habilidades:} atributo que reflete as especialidades, grau de escolaridade, capacidades e competências que não necessariamente devem estar relacionadas ao produto; 

    \item \textbf{tarefas:} aquilo que a persona executa, sendo elas básicas ou críticas. Elas ainda podem ser classificadas quanto a sua frequência, importância e duração;
    
    \item \textbf{relacionamentos:} quem a persona interage. Identificar com quem a persona se relaciona ajuda a mapear outras partes envolvidas ao projeto e que devem ser consideradas durante o processo de design;
    
    \item \textbf{requisitos:} aquilo que as personas precisam, suas necessidades; e
    
    \item \textbf{expectativas:} tanto o que a persona espera do funcionamento do produto, quanto aquilo que é gerado durante seu uso, na sua interação com o produto.
\end{itemize}

Como as personas partem de um processo de investigação das características dos usuários e descrição de seu perfil, a eficiência dessa ferramenta de design está atrelada ao quão próxima a persona se encontra de uma pessoa real e o quanto ela representa o seu público-alvo \cite[p. 154]{BarbosaEtAl2021}. 

Em um projeto podem haver várias personas distintas, que cada qual com seu papel irão auxiliar os designer. Na subseção a seguir este grupo de personas é melhor explorado.

\subsection{Elenco das Personas}

O grupo de personas de um projeto, também chamado de elenco de personas, conta com pelo menos três personas distintas como é definido por \citeonline{Courage_Baxter_2005} e \citeauthor{usability2020}, podendo ser de seis tipos, conforme foi descrito por \citeonline{cooper07}. São estes os tipos de personas:

\begin{itemize}
    \item Primária (\textit{Primary}): esta representa o principal alvo do design. Ela é que tem de ser satisfeita pela interface projetada;
    
    \item Secundária (\textit{Secundary}): esta é satisfeita com a interface projetada para a persona primária, porém a persona secundária necessita de adicionais específicas que podem ser acrescentadas ao design sem prejudicar aquilo que foi projetado para servir à persona primária;
    
    \item Suplementar (\textit{Supplemental}): ela é uma combinação das personas primárias e secundárias. Suas necessidades são completamente representadas por essa combinação de personas;
    
    \item Cliente (\textit{Customer}): esta persona busca atender às necessidades dos clientes, que não necessariamente são dos usuários finais do sistema;
    
    \item Servida (\textit{Served}): este tipo de persona é diferente do tipos de persona já discutidos. Ela não é um usuário do produto; contudo, eles são diretamente afetados pelo uso do produto; e
    
    \item Negativa (\textit{Negative}): chamada também de anti-persona, esta persona é usada para comunicar às partes interessadas e demais \textit{stakeholders} que existem tipos específicos de usuários para os quais o produto não foi projetado. 
\end{itemize}

O elenco de persona é caracterizado por conter ao menos uma persona por papel de usuário sendo que, dentre as personas definidas para cada papel, ao menos uma delas deve ser a persona primária. O ideal é que seja projetada uma interface distinta para cada persona primária \cite[p. 155]{BarbosaEtAl2021}. Por exemplo em um sistema escolar com dois papeis de usuário distintos, como alunos e professores, haveriam pelo menos duas personas primárias (uma representando os professores e outra os alunos), sendo que seria recomendado ser projetado duas interfaces distintas (novamente, uma para os professores e outra para os alunos).

\section{Jogos para Aprendizagem}
\label{sec:jogos-aprend}

Nesta seção é apresentada uma breve contextualização sobre jogos para aprendizagem, além dos elementos que envolvem a construção desse tipo de jogo e como eles se caracterizam. 
\subsection{Contextualização sobre Jogos para Aprendizagem}

A academia tem investigado o uso de abordagens inovadoras para auxiliar o processo de aprendizagem, instigando, atraindo e motivando o aluno no desenvolvimento de atividades pedagógicas \cite{battistella, brito, Sales2020}. Uma das abordagens exploradas é o uso de jogos. No geral, os jogos têm uma capacidade de atrair e motivar as pessoas e de gerar engajamento e dedicação na realização de tarefas \cite{Vianna_Vianna_Medina_Tanaka_Krug_2013}. 

O jogo sério é um tipo de jogo que vai além do entretenimento, o qual pode visar informar, treinar e ensinar. Esse tipo de jogo pode ser aplicado em diversas áreas, tais como saúde, publicidade, política e educação. Jogos sérios também incluem jogos para aprendizagem \cite{Becker_2021}.

No caso de ser aplicado na educação, os jogos para aprendizagem buscam tornar o processo de ensino-aprendizagem mais atrativo e proveitoso, desenvolvendo no estudante habilidades cognitivas através da prática e engajar os alunos nesse processo \cite{sommariva, queiroz, darin}. Dentro da educação existem várias áreas de conhecimento nas quais o jogos são aplicados no intuito de auxiliar o ensino e aprendizagem. Uma dessa áreas é a de Interação Humano-Computador \cite{Sales2020, Sales2020UsoTDS}. 

Jogos para aprendizagem é um tipo de abordagem que vem se tornando cada vez mais popular na educação em computação, pois podem aumentar a eficácia e o engajamento da aprendizagem \cite{battistella, brito, sales_climaco2016, queiroz}. Na subseção a seguir são exploradas as características deste tipo de jogo e o que envolve seu desenvolvimento.

\subsection{Desenvolvimento e Características de Jogos Digitais para Aprendizagem}

Um processo sólido para se desenvolver um jogo que parte da concepção de uma ideia e objetiva chegar em um resultado que promova uma experiência satisfatória ao jogador, é a chave para se construir um \textit{game} \cite[p. 10-11]{Fullerton_2008}.  Segundo \citeonline[p. 10-11]{Fullerton_2008} esta é uma abordagem centrada no jogador, na qual o princípio é envolver o jogador no processo de design do início ao fim, ou seja, manter continuamente a experiência do jogador como o alvo a ser atingido, além de testar a jogabilidade com os jogadores-alvo durante as etapas de desenvolvimento.

O \textit{Playcentric Design Process} (PDP) é um método de desenvolvimento de jogos, que segue esse princípio de envolver o jogador. Nele, o jogador é inserido no processo logo cedo ao serem definidas as metas de experiência do jogador \cite{Fullerton_2008}.

O PDP se baseia num ciclo iterativo, no qual continuamente é prototipado, testado e validado o que foi concebido para o jogo, sempre tendo em vista as metas de experiência \cite[p. 10-11]{Fullerton_2008}. Este método acompanha a visão de IHC que preocupa-se com certos critérios de qualidade da interação entre seres humanos e sistemas digitais \cite[p. 8-10]{BarbosaEtAl2021}.

\citeonline{BarbosaEtAl2021} apontam que "os critérios de qualidade de uso enfatizam certas características da interação e da interface que as tornam adequadas aos efeitos esperados do uso do sistema". São critérios de qualidade, a usabilidade e a experiência do usuário (UX).

\citeonline{Petri_Wangenheim_2019} abordam em seus estudos a questão da qualidade em jogos educacionais e trazem como produto de suas pesquisas um modelo denominado Model for the Evaluation of Educational Games (MEEGA+). Este é um instrumento de avaliação da qualidade de jogos no ensino de informática que contempla fatores de usabilidade e experiência do jogador.

Os fatores de qualidade do MEEGA+ ainda são decompostos em dimensões. Estas dimensões descrevem as características dos jogos educacionais que impactam na qualidade deles \cite{Petri_Wangenheim_2019}.

Outro estudo similar que aborda a caracterização de jogos para aprendizagem é o de \citeonline{deSales_SousaeSilva_2020}. Neste trabalho são contemplados os requisitos de qualidade, a experiência do jogador, além de outras características gerais de jogos sérios no processo e ensino e aprendizagem de IHC. 

Estas características de jogos, tal como suas respectivas relevâncias, foram observadas no estudo de \citeonline{silva_sales_mendes2021} como aspectos de qualidade. Estes aspectos refletem critérios, princípios de \textit{design} \cite{nielsen1994, ISO9126-1, BarbosaEtAl2021}, metas de usabilidade e aspectos de experiência desejáveis do usuário \cite{Preece_Rogers_Sharp_2005} identificados na Engenharia de Software (ES) e em IHC.

Estes aspectos de qualidade são para os designers e desenvolvedores, os insumos fundamentais do processo de desenvolvimento para esse tipo de \textit{software} \cite{silva_sales_mendes2021}. Na Engenharia de Software esta etapa de entendimento inicial dos elementos que envolvem o tema do projeto de um software é denominada de concepção do projeto \cite[p. 127]{Pressman_2000}.

Após a concepção inicial, os aspectos de qualidade ganham forma, sendo especificamente modelados para determinado projeto \cite[p. 151]{BarbosaEtAl2021}. A exemplo de modelagem existem as personas que no caso incorporam estes aspectos de qualidade aos seus atributos \cite[p. 154]{BarbosaEtAl2021}.

No desenvolvimento de um software estes aspectos de qualidade podem ser traduzidos para requisitos não funcionais \cite{swebok2014} e para as metas de experiência do usuário, as quais fazem parte de atividades na engenharia de software, como a prototipação, inspeção, validação e testes \cite{Pressman_2000}.

%\section{Trabalhos Correlatos}
%\label{sec:trab_cor}

% trazer aqui trabalhos sobre a construção de personas, sobre o uso de personas no desenvolvimento de design de interfaces ou de jogos

% pesquisas do projeto de recursos didáticos

% trabalhos sobre personas que saíram agora na conferencia Ibero americana

No Capítulo \ref{chap:Metodo} estão apresentadas a metodologia adotada neste trabalho.