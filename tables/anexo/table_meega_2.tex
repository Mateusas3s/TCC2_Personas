\begin{table}[htbp]
\centering
\caption{Fatores de Experiência do Jogador MEEGA+}
\label{tab:Table_meega_2}
\begin{tabular}{|l|l|l|}
\hline
Dimensão                                & ID & Descrição                                                                                                                                           \\ \hline
Confiança                               & 10 & \begin{tabular}[c]{@{}l@{}}O conteúdo e a estrutura me ajudaram a ter certeza de \\ que aprenderia com este jogo.\end{tabular}                      \\ \hline
Desafio               & 11 & Este jogo é apropriadamente desafiador para mim.                                                                                                    \\ \cline{2-3} 
                                        & 12 & \begin{tabular}[c]{@{}l@{}}O jogo oferece novos desafios (oferece novos obstáculos, \\ situações ou variações) em um ritmo apropriado.\end{tabular} \\ \cline{2-3} 
                                        & 13 & \begin{tabular}[c]{@{}l@{}}O jogo não se torna monótono à medida que avança \\ (tarefas repetitivas ou enfadonhas).\end{tabular}                    \\ \hline
Satisfação            & 14 & \begin{tabular}[c]{@{}l@{}}Concluir as tarefas do jogo me deu uma sensação \\ satisfatória de realização.\end{tabular}                              \\ \cline{2-3} 
                                        & 15 & \begin{tabular}[c]{@{}l@{}}É pelo meu esforço pessoal que consegui avançar no \\ jogo.\end{tabular}                                                 \\ \cline{2-3} 
                                        & 16 & Estou satisfeito com as coisas que aprendi com o jogo.                                                                                              \\ \cline{2-3} 
                                        & 17 & Eu recomendaria este jogo aos meus colegas.                                                                                                         \\ \hline
Interação Social       & 18 & Pude interagir com outros jogadores durante o jogo.                                                                                                 \\ \cline{2-3} 
                                        & 19 & \begin{tabular}[c]{@{}l@{}}O jogo promove a cooperação e / ou competição entre \\ os jogadores.\end{tabular}                                        \\ \cline{2-3} 
                                        & 20 & \begin{tabular}[c]{@{}l@{}}Me senti bem interagindo com outros jogadores durante \\ o jogo.\end{tabular}                                            \\ \hline
Diversão               & 21 & Eu me diverti com o jogo.                                                                                                                           \\ \cline{2-3} 
                                        & 22 & \begin{tabular}[c]{@{}l@{}}Algo aconteceu durante o jogo (elementos do jogo, \\ competição, etc.) que me fez sorrir.\end{tabular}                   \\ \hline
Atenção Focada         & 23 & \begin{tabular}[c]{@{}l@{}}Havia algo interessante no início do jogo que chamou \\ minha atenção.\end{tabular}                                      \\ \cline{2-3} 
                                        & 24 & \begin{tabular}[c]{@{}l@{}}Eu estava tão envolvido em minhas tarefas de jogo que \\ perdi a noção do tempo.\end{tabular}                            \\ \cline{2-3} 
                                        & 25 & Esqueci meu ambiente imediato enquanto jogava.                                                                                                      \\ \hline
Relevância             & 26 & O conteúdo do jogo é relevante para os meus interesses.                                                                                             \\ \cline{2-3} 
                                        & 27 & \begin{tabular}[c]{@{}l@{}}É claro para mim como o conteúdo do jogo se relaciona \\ com o curso.\end{tabular}                                       \\ \cline{2-3} 
                                        & 28 & \begin{tabular}[c]{@{}l@{}}Este jogo é um método de ensino adequado para este \\ curso.\end{tabular}                                                \\ \cline{2-3} 
                                        & 29 & \begin{tabular}[c]{@{}l@{}}Prefiro aprender com este jogo a aprender de outras \\ maneiras (por exemplo, outros métodos de ensino).\end{tabular}    \\ \hline
\begin{tabular}[c]{@{}l@{}}Aprendizagem \\Percebida\end{tabular} & 30 & O jogo contribuiu para o meu aprendizado neste curso.                                                                                               \\ \cline{2-3} 
                                        & 31 & \begin{tabular}[c]{@{}l@{}}O jogo permitiu um aprendizado eficiente em \\ comparação com outras atividades do curso.\end{tabular}                   \\ \hline
\end{tabular}
\legend{Fonte: Adaptado de \cite{Petri_Wangenheim_2019}}
\end{table}