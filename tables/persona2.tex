\begin{table}[htbp]
\centering
\caption{Persona Secundária}
\label{tab:Table_persona2}
\small
\begin{tabular}{| m{0.2\textwidth} m{0.7\textwidth}|}
\hline \multicolumn{2}{|c|}{\textbf{Identidade}} \\ \hline
& \\

\begin{center} 
\includegraphics[scale=0.04]{figuras/personas/model-2911332_1920.jpg} 
Fonte: Pixabay\tablefootnote{https://pixabay.com/photos/model-businessman-corporate-2911332/}
\end{center} 

&

\textbf{Nome: } Afonso Souza de Queiroz

\textbf{Idade:} 19 anos

\textbf{Ocupação:} Estudante de Engenharia de Software na UnB - Gama

\\ \hline


\multicolumn{2}{|c|}{\textbf{Descrição}} \\ \hline
\multicolumn{2}{|p{15cm}|}{
     \begin{tabular}[c]{@{}l@{}}\\
        \textbf{Já joguei} alguns jogos para \textbf{aprender um conteúdo} e que me possibilitava \textbf{avaliar} se \\eu realmente tinha aprendido. Eu jogava com certa \textbf{moderação}. Atualmente eu \textbf{não uso} \\\textbf{jogos} desse tipo, mas o motivo é porque eu \textbf{alcancei meus objetivos} de estudo e por\\ enquanto não encontrei nenhum outro jogo que me auxilie. Estou \textbf{cursando IHC}, por\\ isso \textbf{não tenho muito conhecimento} em relação a elaboração de design de interfaces,\\ pois o pouco que tenho veio apenas de projetos de \textbf{outras disciplinas}. Estudo mais \\ sozinho, \textbf{pesquisando na internet} e buscando por conta própria materiais de estudo. \\Apenas quando não encontro, reviso os \textbf{materiais} \textbf{disponibilizados pelo professor} e \\também não descarto a ajuda de \textbf{colegas, monitor} \textbf{da disciplina e do professor}.\\
        \\
        Para mim, um jogo onde vou aprender deve ter um design com um \textbf{padrão simples},\\ mas que desperte \textbf{interesse}; deve me dar bons \textbf{feedbacks} a cada interação que faço;\\ deve ser algo \textbf{simples de aprender a jogar} e \textbf{simples de jogar}, nada muito difícil; e \\seria bom que os \textbf{textos do conteúdo} fossem bem objetivos, com \textbf{cores e fontes} bem\\ ajustadas para a leitura. Espero ter a sensação logo de cara que \textbf{não vou perder meu} \\\textbf{tempo} com o jogo, mas sim que vou \textbf{perceber o quanto o conteúdo é importante} e \\o quanto foi \textbf{satisfatório} ter me \textbf{divertido} aprendendo. E espero também de certa ter \\pequenos \textbf{desafios}, pois isso me ajuda a \textbf{focar} na atividade que estou realizando.\\ \\ 
    \end{tabular}
} \\ \hline
\end{tabular}
\legend{Fonte: Própria Autoria}
\end{table}