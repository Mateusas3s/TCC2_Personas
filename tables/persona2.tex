\begin{table}[htbp]
\centering
\caption{Persona Secundária}
\label{tab:Table_persona2}
\small
\begin{tabular}{| m{0.25\textwidth} m{0.65\textwidth}|}
\hline \multicolumn{2}{|c|}{\textbf{Identidade}} \\ \hline
& \\

\begin{center} 
\includegraphics[scale=0.06]{figuras/personas/model-2911332_1920.jpg} 
Fonte: Pixabay\tablefootnote{https://pixabay.com/photos/model-businessman-corporate-2911332/}
\end{center} 

&

\textbf{Nome: } Afonso Souza de Queiroz

\textbf{Idade:} 19 anos

\textbf{Ocupação:} Estudante de Engenharia de Software na UnB, Faculdade do Gama

\\ \hline


\multicolumn{2}{|c|}{\textbf{Descrição}} \\ \hline
\multicolumn{2}{|p{15cm}|}{
     \begin{tabular}[c]{@{}l@{}}\\
        Já joguei alguns jogos educacionais onde meu principal objetivo era aprender um\\ conteúdo novo. Além disso era interessante quando o jogo possibilitava que eu revisasse\\ o conteúdo e avaliasse o que tinha aprendido. Eu jogava com certa moderação. Por mais\\ que atualmente eu não esteja usando algum jogo educacional, eu parei de jogar pois \\alcancei meus objetivos de estudo e me sentiria da mesma forma se houvesse algum\\ jogo que me ajudasse nas matérias da faculdade.\\ 
        \\
        Ainda não fiz a disciplina de IHC, por isso não tenho muito conhecimento em relação\\ a elaboração de design de interfaces, sendo que o pouco que tenho veio de projetos de\\ outras disciplinas e atividades extra curriculares.\\
        \\
        Gosto mais de estudar sozinho, pesquisando na internet e indo por conta própria atrás\\ de materiais de estudo. Apenas quando não acho nada relevante, recorro aos materiais\\ disponibilizados pelo professor.\\
        \\
        Para mim, um jogo onde vou aprender deve ter um design com um \textbf{padrão simples},\\ mas que desperte \textbf{interesse}; deve me dar bons \textbf{feedbacks} a cada interação que faço\\ no jogo; deve ser algo \textbf{simples de aprender como jogar} e \textbf{simples de jogar}; e caso\\ tenha textos sobre o conteúdo, seria bom que os textos fossem bem objetivos e com\\ uma cor e fonte que facilitasse a leitura.\\ 
        \\
        Espero ter a sensação logo de cara que \textbf{não vou perder meu tempo} com o jogo, mas\\ que ao final vou perceber o quanto foi \textbf{importante aprender o conteúdo} e o quanto\\ foi \textbf{satisfatório} ter me \textbf{divertido} aprendendo. E espero também de certa forma ser\\ \textbf{desafiado}, pois isso me ajuda a \textbf{focar} na atividade que estou realizando.\\ \\ 
    \end{tabular}
} \\ \hline
\end{tabular}
\end{table}
\newpage